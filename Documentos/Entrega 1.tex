\documentclass[11pt,letterpaper]{article}
\usepackage[spanish]{babel} %idioma
\usepackage[latin1]{inputenc}
\usepackage{graphicx}
\usepackage{anysize} %Permitir distintas medidas de margenes

% Configuraci�n PDF
\usepackage[pdftex]{hyperref}
\hypersetup{
    %bookmarks=true,
    pdfborder={0 0 0},
    pdfstartview=FitH,
    pdftitle={Arquitectura de computadores},
    pdfsubject={Entrega 1},
    pdfauthor={G11}
}

\marginsize{3cm}{2cm}{2cm}{4cm}

% $Rev: 595 $
\title{\Huge \textbf{Arquitectura de computadores}\\\Large \textbf{Entrega 1}}
\author{Grupo G11}
\date{ de 2009}
\begin{document}

\maketitle

\newpage
%Tabla de contenidos y figuras. Recomendamos no tocar
\tableofcontents
\listoffigures

\newpage
\section{Descripci\'on General}
\subsection{Perspectiva del Producto}
	%�Qu� espera el cliente del producto? �C�mo lo va a ayudar? �Qu� problema va a solucionar?
	Woodtech es una empresa que ofrece servicios de mediciones de diversos tipos, aplicadas actualmente a las industrias madereras y mineras. Los productos ofrecidos constan, a grandes rasgos, de elementos f�sicos (\textit{Hardware}), como el portal de mediciones y diversos servidores; y por otro lado de Software, como algoritmos de c�lculo e interfaces de auditor�a y de control de mediciones.
	
	El control de mediciones se realiza a trav�s de la aplicaci�n denominada \textit{Workbench}, cliente que consume servicios de servidores (particularmente del \textit{Engine}). Este permite a los operarios capturar, seleccionar, revisar y modificar mediciones sobre transportes de un determinado tipo de material como madera o carb�n. 
	
	As�, en el contexto de renovaci�n y evoluci�n de la empresa, se buscar� solucionar los problemas que presenta el actual \textit{Workbench} utilizado por los operarios. Este \textit{Workbench} fue pensado para ser usado �nicamente en pruebas internas, pero luego se adapt� como un producto final. Adicionalmente, el software actual est� implementado usando \textit{CodeGear Delphi}; sin embargo, debido a que el resto de los programas de Woodtech como el \textit{Engine} usan Java, es deseable que el software tambi�n est� implementado usando esta plataforma.	Debido a estas razones, el software actualmente no satisface las necesidades de extensibilidad y mantenibilidad requeridas por el cliente.
	
	De esta forma, el producto a desarrollar ser� un reemplazo completo al Workbench ya existente desarrollado por el cliente, y tendr� su misma funcionalidad. Buscar� solucionar estas necesidades, entregando un software escrito en el lenguaje Java con un dise�o que le d� extensibilidad y mantenibilidad suficiente para facilitar el agregamiento de nuevas capacidades. Adem�s, se construir� de forma que sea mantenible por el equipo de desarrollo del cliente. Por otra parte, deber� proveer una interfaz amigable que sea posible de ocupar por un operario.

\end{document}
