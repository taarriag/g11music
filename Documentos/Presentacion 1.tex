%----------------------------------------------
%                  1 Preamble
%----------------------------------------------
%----------------------------------------------
%1.1 Beamer Preamble
%----------------------------------------------
\documentclass[11pt,slidestop,compress,mathserif]{beamer}
\usetheme{Malmoe} % Beamer theme
\usecolortheme{lily} % Beamer color theme
%----------------------------------------------
%1.2 LaTeX Preamble
%----------------------------------------------
\usepackage[spanish]{babel} %idioma
\usepackage[latin1]{inputenc}
\usepackage{graphicx}
%----------------------------------------------
%                    2 Body
%----------------------------------------------
%----------------------------------------------
%2.1 Title Page
%----------------------------------------------
\title{Arquitectura de Computadores: Entrega 1}
\author{Grupo 11}
\institute{Pontificia Universidad Catolica de Chile}
\begin{document}
\begin{frame} % Cover slide
\titlepage
\end{frame}
%----------------------------------------------
%2.2 Contents
%----------------------------------------------
\section{SmartMusic} % Bookmark information
\subsection{Descripci�n del Proyecto} % Bookmark information
%----------------------------------------------
\begin{frame}
	\pause \frametitle{Descripci�n del Proyecto}
	\begin{itemize}[<+->]
		\item Nuestro dispositivo consistir� en un perif�rico para computador, cuya caracter�stica principal es la de controlar autom�ticamente la m�sica que el computador se encuentra reproduciendo y su volumen seg�n condiciones ambientales. 
		\item Para esto, el dispositivo deber� monitorear el ambiente con el fin de detectar cambios en la intensidad de luz y del ruido ambiental, regulando el tipo de m�sica que se reproducir� y su volumen.
		\begin{itemize}[<+->]
			\item Por ejemplo, si la luz ambiental es fuerte se tocar� m�sica movida, y si el ruido ambiental es bajo el volumen tambi�n lo ser�.
	\end{itemize}
	\item De esta forma, nuestro dispositivo captar� en cierta manera la vibra del ambiente y as� reproducir� la m�sica adecuada.
\end{itemize}
\end{frame}
%-----------------------------------------------
\begin{frame}
\frametitle{Descripci�n del Proyecto}
	\begin{itemize}[<+->]
		\item El dispositivo dispondr� de una pantalla LED que mostrara informacion de la cancion en reproducci�n, como Titulo, autor, album etc...
		\item El usuario podr� programar un tiempo en que el dispositivo pueda iniciar o detener autom�ticamente la reproducci�n.
		\item Adicionalmente, el dispositivo permitir� comenzar y pausar la reproducci�n cuando el usuario lo desee, como tambi�n adelantar y retroceder de canci�n. 
		\item Las canciones que se reproducir�n para las distintas vibras presentes en el ambiente se podr�n configurar desde el computador.	
	\end{itemize}
\end{frame}
%-----------------------------------------------
\end{document}
%-----------------------------------------------