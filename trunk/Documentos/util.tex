% $Id: util.tex 708 2008-03-24 00:06:12Z jpcanepa $
\usepackage{ulem}			% For strikethrough text
\usepackage{listings}	% Code typessting
\usepackage{ifthen}		% For \ifthenelse
\usepackage{graphicx}

\usepackage[spanish]{babel} %idioma
\usepackage[latin1]{inputenc}

% Counters
\newcounter{lecturenumber}		% Lecture number

%%%% LISTINGS CUSTOMIZATION %%%%%%%%%%%%%%%%%%%%%%%%%%%%%%%%%%%%%%%%%%%%%%%%%%%

% Color de los comentarios, por si escapamos a LaTeX en un lstlisting
\newcommand{\CommentColor}{\color{green!60!black}}

% String color used in listings
\newcommand{\StringColor}{\color{red!70!black}}

% User keywords color
\newcommand{\UserKeywordsColor}{\color{cyan!50!black}}

% Keywords color
\newcommand{\KeywordsColor}{\color{blue}}

% Define the C# languaje style
\lstdefinelanguage{CSharp}
{
	basicstyle=\small\ttfamily,
	keywordstyle=\KeywordsColor\textbf,
	keywordstyle=[2]\UserKeywordsColor,
	keywordstyle=[3]\StringColor,
	tabsize=2,
	morekeywords={abstract, as, base, bool, break, byte, case, 
		catch, char, checked,class, const, continue, decimal, 
		default, delegate, do, double, else, enum, event, explicit,
		extern, false, finally, fixed, float, for, foreach, get, goto, 
		if, implicit, in, int, interface, internal, is, lock, long,
		namespace, new, null, object, operator, out, override, 
		params, partial, private, protected, public, readonly, ref, return, 
		sbyte, sealed, set, short, sizeof, stackalloc, static, string,
		struct, switch, this, throw, try, typeof, true, uint, ulong, 
		unchecked, unsafe, ushort, using, value, virtual, volatile,
		void, while, where},
	morekeywords=[2]{Program,Console,String},
	morekeywords=[3]{@}, 
	commentstyle=\CommentColor,
	stringstyle=\StringColor,
	sensitive=true,
	morecomment=[l]{//},
	morecomment=[s]{/*}{*/},
	morestring=[b]",
	showstringspaces=false,
	aboveskip=0pt, 
	belowskip=0pt,
	mathescape=true
}

\lstdefinelanguage{BASIC}
{
	mathescape=false,
	basicstyle=\small\ttfamily,
	keywordstyle=\KeywordsColor\textbf,
	keywordstyle=[2]\UserKeywordsColor,
	tabsize=2,
	keywords={DATA,IF,RANDOMIZE,STOP,DIM,INPUT,READ,THEN,END,LET,REM,
		TO,FOR,NEXT,RESTORE,TROFF,GOSUB,ON,RETURN,TRON,GOTO,PRINT,
		STEP,AND,OR},
	keywords=[2]{LEN,LEFT\$,RND,INT,SIN,COS,TAN,ATN,SQR,MAX,MIN,ABS,
		LEFT\$,RIGHT\$,MID\$,CHR\$,LEN,VAL,SPC\$,LOG,FRE,SGN,
		TAB,STR\$},
	commentstyle=\CommentColor,
	stringstyle=\StringColor,
	sensitive=true,
	morestring=[b]",
	showstringspaces=false,
	aboveskip=0pt, 
	belowskip=0pt,	
}

\lstdefinelanguage{Java2}[]{Java}
{
	keywordstyle=[2]\UserKeywordsColor
}

% Set as the default languaje
\lstset{language=CSharp}

%%%%% CONVENIENCE COMMANDS %%%%%%%%%%%%%%%%%%%%%%%%%%%%%%%%%%%%%%%%%%%%%%%%%%%%

% Allow the user to add more user-level keywords
\newcommand{\AddUserKeywords}[1]{\lstset{morekeywords=[2]{#1}}}

% Allow the user to change the size the code is typeset with (without knowing the 
% exact lstsset syntax
\newcommand{\CodeSize}[1]{\lstset{basicstyle=#1\ttfamily}}

% Outputs title slide information
% @param Lecture number
% @param Main title
% @param Subtitle
% @param Lecture date
\newcommand{\presentationheader}[4]
{
\setcounter{lecturenumber}{#1}
\title[Entrega \thelecturenumber]{{\scriptsize IIC2342 -- Arquitectura de Computadores} \\ #2}
\author[G11]{Grupo 11}
\subtitle{#3}
\date{#4}
\institute[PUC]{Pontificia Universidad Cat\'olica de Chile}
}

% Use to typeset placeholder text that must be replaced before 
% final release
\newcommand{\placeholder}[1]{{\color{red}\fbox{#1}}}

%%%%% BEAMER CUSTOMIZATION %%%%%%%%%%%%%%%%%%%%%%%%%%%%%%%%%%%%%%%%%%%%%%%%%%%%

% By default, use <+-> as the overlay specification
\beamerdefaultoverlayspecification{<+->}

% Main theme
\usetheme[secheader]{Boadilla}

% Let math be typeset using serif fonts
\usefonttheme[onlymath]{serif}

% At the beginning of every subsection, display a single slide with the 
% section and subsection title
\AtBeginSubsection[]
{
\begin{frame}
	\begin{center}
		{ \usebeamercolor[fg]{frametitle}
			{\scriptsize \insertsection}  \\
			{\Huge \insertsubsection}
		}
	\end{center}
\end{frame}
}

% At the beginning of each section, output a slide with the section title
\AtBeginSection[]
{
\begin{frame}
	\begin{center}
		{ \usebeamercolor[fg]{frametitle}
			{\Huge \insertsection}
		}
	\end{center}
\end{frame}
}

% Replace those ugly looking balls with some slightly less ugly circles
% in enumerate/itemize and TOC
\setbeamertemplate{enumerate items}[circle]
\setbeamertemplate{itemize items}[circle]
\setbeamertemplate{sections/subsections in toc}[circle]

% Set the footline to something simpler. No need for Name/Lecture, etc. Just a plain 
% color box with the slide counter and the lecture number
\setbeamertemplate{footline}{
  \leavevmode%
  \hbox{%
  \begin{beamercolorbox}[wd=0.15\paperwidth,ht=2.25ex,dp=1ex,left]{title in head/foot}%
    \centering\usebeamerfont{title in head/foot} \insertshorttitle
  \end{beamercolorbox}}%
  \hbox{%
  \begin{beamercolorbox}[wd=0.85\paperwidth,ht=2.25ex,dp=1ex,right]{author in head/foot}%
    \usebeamerfont{author in head/foot} \insertframenumber{} / \inserttotalframenumber\hspace*{2ex} 
  \end{beamercolorbox}}%
  \vskip0pt%
}

% Supress navigation symbols
\setbeamertemplate{navigation symbols}{}

% Redefine the "example" environment
% @param The title of the example
%  - Adds a header with the example number in the format "<lecture number>.<example number>"
\renewenvironment{example}[1]{%
	\begin{exampleblock}{{\small {\bf Ejemplo}: #1}}
}{
	\end{exampleblock}
}
