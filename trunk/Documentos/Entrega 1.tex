\documentclass[11pt,letterpaper]{article}
\usepackage[spanish]{babel} %idioma
\usepackage[latin1]{inputenc}
\usepackage{graphicx}
\usepackage{anysize} %Permitir distintas medidas de margenes

% Configuraci�n PDF
\usepackage[pdftex]{hyperref}
\hypersetup{
    %bookmarks=true,
    pdfborder={0 0 0},
    pdfstartview=FitH,
    pdftitle={Arquitectura de computadores},
    pdfsubject={Entrega 1},
    pdfauthor={G11}
}

\marginsize{3cm}{2cm}{2cm}{4cm}

% $Rev: 595 $
\title{\Huge \textbf{Arquitectura de Computadores}\\\Huge \textbf{Nombre del Proyecto}\\\Large \textbf{Entrega 1}}
\author{Grupo G11}
\date{24 de Agosto de 2009}
\begin{document}

\maketitle

\bigskip 
\textbf{Integrantes:}
\begin{itemize}
\item Pablo Alcayaga
\item Tom�s Arriagada
\item Felipe Balbont�n
\item Gabriel Di�guez
\item Javier Espinosa
\item Sebasti�n Vicencio
\end{itemize}

\newpage
%Tabla de contenidos y figuras. Recomendamos no tocar
\tableofcontents
\listoffigures

\newpage
\section{Descripci�n del proyecto}
\subsection{Descripci�n general}
El dispositivo a realizar consiste en un controlador de reproducci�n de m�sica. B�sicamente, el controlador recibir� informaci�n del ambiente en que se encuentra, y en base a �sta controlar� diversos aspectos de la reproducci�n.

\subsection{Funcionalidad}
Nuestro dispositivo consistir� en un perif�rico para computador, el cual controlar� tanto el volumen de la reproducci�n como la m�sica que se toca. Para esto, debe poder sensar el ambiente con el fin de detectar cambios en la intensidad de luz y del ruido ambiental, regulando el tipo de m�sica y el volumen de �sta respectivamente.
Por ejemplo, si la luz ambiental es fuerte se tocar� m�sica movida, y si el ruido ambiental es bajo el volumen tambi�n lo ser�. De esta forma, nuestro dispositivo captar� en cierta manera la \textit{vibra} del ambiente y as� reproducir� la m�sica adecuada.

\subsection{Caracter�sticas}

 
\section{Dispositivos a utilizar}
\begin{table}[h]
	\begin{center}
	\begin{tabular}{|c|c|c|c|c|p{4.0cm}|}
		\hline  Nombre & C�digo  & Disponibilidad & Precio & Detalles & Uso \\ 
		\hline  Microcontrolador PIC & PIC16F877A & \href{http://www.olimex.cl/product_info.php?products_id=159}{Olimex.cl}  & \verb@$@4,583 & \href{http://ww1.microchip.com/downloads/en/DeviceDoc/30292c.pdf}{datasheet} & Permite ejecutar las instrucciones asociadas a cambios en la luminosidad, el ruido del ambiente y el input del usuario.\\
		\hline  16x2 Character LCD & GDM1602K & \href{http://www.olimex.cl/product_info.php?products_id=154}{Olimex.cl} & \verb@$@7,668 & \href{http://www.olimex.cl/pdf/LCD/GDM1602K.pdf}{datasheet} & Muestra informaci�n de la canci�n actual y del estado del ambiente. \\

		\hline  Sensor de Luz & ZX-LDR & \href{http://www.olimex.cl/product_info.php?cPath=50_89&products_id=340}{Olimex.cl} & \verb@$@3,934 & \href{http://www.olimex.cl/pdf/ZX-LDR.pdf}{datasheet} & Detecta los niveles de luminosidad del ambiente. \\ 

		\hline  Sensor de sonidos & ZX-Sound & \href{http://www.olimex.cl/product_info.php?cPath=50_95&products_id=408}{Olimex.cl} & \verb@$@4,900 & \href{http://www.olimex.cl/pdf/INEX/ZX-sound_e.pdf}{datasheet} & Detecta el nivel de ruido ambiental. \\
		
		\hline
	
	\end{tabular}
	\end{center}
\end{table}

\end{document}
