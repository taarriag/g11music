\documentclass[11pt,letterpaper]{article}
\usepackage[spanish]{babel} %idioma
\usepackage[latin1]{inputenc}
\usepackage{graphicx}
\usepackage{anysize} %Permitir distintas medidas de margenes

% Configuraci�n PDF
\usepackage[pdftex]{hyperref}
\hypersetup{
    %bookmarks=true,
    pdfborder={0 0 0},
    pdfstartview=FitH,
    pdftitle={Arquitectura de computadores},
    pdfsubject={Entrega 1},
    pdfauthor={G11}
}

\marginsize{3cm}{2cm}{2cm}{4cm}

% $Rev: 595 $
\title{\Huge \textbf{Arquitectura de computadores}\\\Large \textbf{Entrega 1}}
\author{Grupo G11}
\date{24 de Agosto de 2009}
\begin{document}

\maketitle

\newpage
%Tabla de contenidos y figuras. Recomendamos no tocar
\tableofcontents
\listoffigures

\newpage
\section{Descripci�n del proyecto}
\subsection{Descripci�n general}
El dispositivo a realizar consiste en un controlador de reproducci�n de m�sica. B�sicamente, el controlador recibir� informaci�n del ambiente en que se encuentra, y en base a �sta controlar� diversos aspectos de la reproducci�n.

\subsection{Funcionalidad}
Nuestro dispositivo controlar� tanto el volumen de la reproducci�n como la m�sica que se toca. Para esto, debe poder sensar el ambiente con el fin de detectar cambios en la intensidad de luz y del ruido ambiental, regulando el tipo de m�sica y el volumen de �sta respectivamente.
Por ejemplo, si la luz ambiental es fuerte se tocar� m�sica movida, y si el ruido ambiental es bajo el volumen tambi�n lo ser�. De esta forma, nuestro dispositivo captar� en cierta manera la \textit{vibra} del ambiente y as� reproducir� la m�sica adecuada.

\subsection{Caracter�sticas}

 
\section{Dispositivos a utilizar}


\end{document}
